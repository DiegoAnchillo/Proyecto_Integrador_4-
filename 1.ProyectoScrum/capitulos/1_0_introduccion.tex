\chapter{Introducción}

\section{Propósito}
Existen muchos restaurantes de comida rápida dentro de distintos centros comerciales, los cuales para hacer un pedido se deben hacer numerosas colas en sus respectivas cajas. Este sistema digital, ayudará a minimizar las colas y hacer un pedido desde el móvil del cliente, el cual estará sentado esperando su pedido.

\section{Alcance}
El alcance de este proyecto esta dirigido a un restaurante del Jockey Plaza ubicado en Santiago de Surco.

\section{Definiciones, siglas y abreviaturas}
\begin{itemize}
	\item Http:Abreviatura de la forma inglesa Hypertext Transfer Protocol, ‘protocolo de transferencia de hipertextos’, que se utiliza en algunas direcciones de internet.
	\item Android:Android es el nombre de un sistema operativo que se emplea en dispositivos móviles, por lo general con pantalla táctil. De este modo, es posible encontrar tabletas (tablets), teléfonos móviles (celulares) y relojes equipados con Android
	\item Web:Conjunto de información que se encuentra en una dirección determinada de internet.
	\item Java:Java es un lenguaje de programación orientado a objetos que se incorporó al ámbito de la informática en los años noventa.
	\item Python:Python es un lenguaje de programación interpretado cuya filosofía hace hincapié en una sintaxis que favorezca un código legible
	
\end{itemize}

\section{Referencias}